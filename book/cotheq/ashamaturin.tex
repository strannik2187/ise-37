Моё эссе о том, <<почему я пошел в IT?>>. Честно скажу, где-то лет 5 назад, я еще не думал, что я пойду в IT. Я учился в техникуме 4 года, на вентиляционщика. В течении всего обучения в своем колледже, я понял, что на ту специальность которую я учусь, это не совсем мое. То есть, это не в том смысле, что у меня не получалось выполнить какую то работу. А дело в том, что я не чувствовал от этой работы какое-то удовольствие, моя бывшая специальность, является так-же востребовательная, и каждый работодатель нуждается в таком рабочем. Но все равно, как бы наверное это странно не звучало, это не мое, и я не почувствовал ничего, что меня могло бы привлечь туда. А на втором курсе, когда у меня сломался компьютер по не понятным причинам, я решил разобрать и посмотреть в чем там дело, скажу кратко, дело было не системном блоке, а именно проблема с системой. После чего я стал искать много информации в интернете по моей проблеме, и пытался разобраться. Когда нашел все таки подходящую информацию по моей теме, я и приступил к своей работе. И мне пришлось, работать, с командной строкой, и прочими функциями, чтобы после выполнения некоторых операций, компьютер ожил, и я смог дальше восстановить его работу. И после этого момента, я понял, что то что я делал, это было очень интересно, и возможно полезно, я получил от этого удовольствие. Это оказалось, то что мне нужно.

В дальнейшем пока я обучался в техникуме, дополнительно, учил для себя программирование искал информацию о компьютерах, и системах. Искал на чём вообще разрабатывались те или иные программы. И это было очень интересно, потому что программист изучающий языки программирования может:
\begin{itemize}
	\item{создавать какую-либо программу}
	\item{решать математические, и логичесие задачи}
\end{itemize}
и т.д
После не большого ознакомления, решил сам изучить какой-нибудь простой для начинающего человека язык программирования. Мои первым языком стал PascalABC.net, на нем я писал математические примеры, выводя решение на экран. После попробовал создать какую нибудь программу, и по скольку я люблю поиграть иногда в игры, написал игру <<Змейку>> используя обучающее видео по созданием игр на Pascal, и статьи. Не скажу что это очень легко, потому что если сильно не всматриваться в общую структуру разработки игр или решению какой-нибудь задачи, то это с виду кажется очень легко и просто.
Например, я чут-ли не откаждого третьего человека слышал такие слова:

\textit{<<Да, это все легко, просто взял нарисовал рисунок, создал персонажа(как змейка), написал движение, нарисовал стенку, и нарисовал фрук, чтобы он появлялся в разных местах всегда, и все.>>}

Это с виду снова повторюсь кажется легко, но это не так, надо работать над написанием кодом и изучать. Чтобы например создать графическое окно, и создать в этом окне клетки по которым будет двигаться змейка, создать сам вид змейки, и т.д.

В общем написание программы это не легкое задание. И тут надо трудиться, чтобы все получилось. Скажу честно, мне не дается все это легко и просто, но это занятие мене все равно очень нравится.

И после обучения в техникуме, я сразу подавал документы на поступление в университет.

Мой выбор --- IT-специалист.

IT-специалист --- профессия очень востребовательная и нужная. Такие специалисты нужны в любой компании, независимо от масштаба деятельности, так как в России появляется много заграничных фирм.

Многие привыкли к тому, что выпускники школ выбирают довольно типичные профессии, такие как:
\begin{itemize}
\item врач
\item парикмахер
\item риэлтор
\item адвокат
\item учитель
\item бухгалтер
\end{itemize}
и другие…
Более того, многие и не подозревают о существовании более интересных профессий, связанных с различными сферами деятельности. Конечно, все перечисленные мной профессии незаменимы в нашей жизни, но, если человек не плохо разбирается допустим в программировании, почему бы не попытаться попробовать свои силы в профессиях более тяжелых умственных?

Так я и поступил. Мои родители довольны моим выбором профессии. Я считаю, что каждый должен выбирать то, что ему близко, и интересно.

Мне близка работа с компьютером.

Во-первых компьютеры в современном мире стали --- незаменимой вещью, без которой уже трудно жить.
Во вторых, сейчас не много людей, которые могут работать с компьютером.
И если очень стараться то можно многого добиться в жизни.

Окончательный ответ на вопрос <<Почему я пошел в IT?>>

Ко всему написанном, я бы хотел дать окончательный и уже короткий ответ на вопрос <<Почему я пошел в IT?>>, а ответ очень прост. Мне очень нравится заниматься за компьютером, я получаю от этого во-первых удовольствие, во-вторых получаю знания, навыки и умения, которые я могу применять в будущем. И именно по этому я пошел в IT.

\noindent Мини цитата
\textit {<<Ну как то так!>>}